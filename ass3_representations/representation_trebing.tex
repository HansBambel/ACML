\documentclass[a4paper]{article}
\usepackage{amsmath}
\usepackage{multicol}

\title{Advanced Concepts of Machine Learning: Representations}
\author{Kevin Trebing (i6192338)}

\begin{document}
\maketitle

\section{Representation for example 1 (Movie theatre): Relational representation}
In this example a relational representation is the most feasible, since a classification for being a good movie can be dependent on many different features. These features can be stored efficiently using multiple tables. Since the cast can be mapped from one movie to multiple actors a multi-instance approach would result in multiple table entries for one movie where only the actor name changes. Using the following relations is more efficient:
movieInfo(movie, directorName, genre, age-restriction, reboot), director(movie, directorName, DirectorOscars), movieActor(movie, name), actor(name, actorOscars), year(movie, year). The return value is the box office result.

\begin{itemize}
\item $bor(M, 28mio)$ :- $movieInfo(M, DN, $'action'$, AR, true), director(M, DN, DO), $

\indent\indent\indent\indent$movieActor(M, N), actor(N, AO), genre(M, $'action'$, AR), sum(DO, AO) > 2$

In natural language: A movie's box office result is \$28 million if the genre is 'action', the movie is a reboot and the the number of Oscars in total is greater than 2.

\item $bor(M, 15mio)$ :- $movieInfo(M, DN, $'comedy'$, $'restricted'$, R), $

 
\indent\indent\indent\indent$movieActor(M, $'Ryan Reynolds'$), year(M, Y), Y > 2012$

In natural language: A movie's box office result is \$15 million if it is R-rated and has 'Ryan Reynolds' as an actor and the movie came out later than 2012.

\item $bor(M, 80mio)$ :- $director(M, $'Christopher Nolan'$, DO)$

This example should be self-explanatory.
\end{itemize}

This approach is also useful for updating the database. If an actor wins an oscar only the entry in the actor table needs to be updated once. In a multi-instance representation all entries of the actor need to be changed.

\end{document}
