\documentclass[a4paper]{article}
\usepackage{amsmath}
\usepackage{multicol}

\title{Advanced Concepts of Machine Learning: Representations}
\author{Kevin Trebing (i6192338)}

\begin{document}
\maketitle

\section{Representation for example 1 (Movie theatre): Relational representation}
In this example a relational representation is the most feasible, since a classification for being a good movie can be dependent on many different features. These features can be stored efficiently using multiple tables. Since the cast can be mapped from one movie to multiple actors a multi-instance approach would result in multiple table entries for one movie where only the actor name changes. Using the following relations is more efficient:
movieInfo(movie, directorName, genre, age-restriction, reboot, box-office-result), director(movie, directorName, DirectorOscars), movieActor(movie, name), actor(name, actorOscars), year(movie, year).

\begin{itemize}
\item $pos(M)$ :- $movieInfo(M, \_, $'action'$, \_, true, BOR), director(M, \_, DO), $

\indent\indent\indent\indent$movieActor(M, N), actor(N, AO), sum(DO, AO) > 2, BOR > \$28 mio$

In natural language: A movie is profitable if the box office result was bigger than \$28 million, the genre is 'action', the movie is a reboot and the the number of Oscars in total is greater than 2.

\item $pos(M)$ :- $movieInfo(M, \_, $'comedy'$, $'restricted'$, \_, \_), $

 
\indent\indent\indent\indent$ movieActor(M, $'Ryan Reynolds'$), year(M, Y), Y > 2012$

In natural language: A movie is profitable if it is R-rated and has 'Ryan Reynolds' as an actor and the movie came out later than 2012.

\item $pos(M)$ :- $director(M, $'Christopher Nolan'$, _)$

This example should be self-explanatory.
\end{itemize}

This approach is also useful for updating the database. If an actor wins an oscar only the entry in the actor table needs to be updated once. In a multi-instance representation all entries of the actor need to be changed.

\section{Representation for example 4 (Spam or ham): attribute valued representation}
For this example we use an attribute valued representation. For this we can use a table for multiple examples:

\begin{tabular}{c|c|c||c}
	subject & knownSender & domain & class \\
	\hline
	supplements & no & unknown & spam \\
	money & yes & bank & ham \\
	unknown & yes & gmail & ham \\
	money & no & unknown & spam \\
	course & no & university & ham \\
	supplements & no & university & ham \\
	arabian prince & no & hotmail & ham \\

\end{tabular}


Possible hypthesis:
\begin{itemize}
\item $ham$ :- $spamorham(\_, \_, university) $
\item $ham$ :- $spamorham(\_, yes, \_, \_)$
\item $spam$ :- $spamorham($'arabian prince'$, no, \_)$
\end{itemize}

We could also use an boolean representation, but then we would have to force all our attributes where we have more than two values into a boolean representation which could lead to a combinatorial explosion. If we reduce the granularity of the attributes to just boolean representations we could lose information.

Using a multi-instance representation would not give us much of a help here, since

\end{document}
