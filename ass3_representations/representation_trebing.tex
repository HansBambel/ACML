\documentclass[a4paper]{article}
\usepackage{amsmath}
\usepackage{multicol}
\usepackage{graphics}

\title{Advanced Concepts of Machine Learning: Representations}
\author{Kevin Trebing (i6192338)}

\begin{document}
\maketitle

\section{Representation for example 1 (Movie theatre): Relational representation}
In this example a relational representation is the most feasible, since a classification for being a good movie can be dependent on many different features. These features can be stored efficiently using multiple tables. 

%movieInfo: \\
\resizebox{\columnwidth}{!}{
\begin{tabular}{|c|c|c|c|c|c|}
\hline
movie & directorName & genre & age-restriction & year & oscars \\
\hline
Interstellar & Christopher Nolan & Adventure & PG-13 & 2014 & 1 \\
Inception & Christopher Nolan & Action & PG-13 & 2010 & 4 \\
LotR: Return of the King & Peter Jackson & Action & PG-13 & 2003 & 11 \\
\hline
\end{tabular}
}

\smallskip
\resizebox{\columnwidth}{!}{
%actor: \\
\begin{tabular}{|c|c|}
\hline
actorName & oscars \\
\hline
Leonardo DiCaprio & 1 \\
Matthew McConaughey & 1 \\
Elijah Wood & 0 \\
Tom Hardy & 0 \\
\hline
\end{tabular}
\quad
%movieActor: 
\begin{tabular}{|c|c|}
\hline
movie & actor \\
\hline
Interstellar & Matthew McConaughey \\
Inception & Leonardo DiCaprio \\
Inception & Tom Hardy \\
LotR: Return of the King & Elijah Wood \\
\hline
\end{tabular}
}

\smallskip
%director: \\
\begin{tabular}{|c|c|}
\hline
directorName & oscars \\
\hline
Christopher Nolan & 0 \\
Peter Jackson & 3 \\
\hline
\end{tabular}

\medskip
Hypothesis language:
\begin{itemize}
\item pos(M) :- movieInfo(M, DN, \_, \_, \_, \_), movieActor(M, AN), actor(AN, AO), director(DN, DO), sum(AO, DO) $>$ 2

In natural language: A movie is profitable if the amount of oscars of the director and at least one actor is bigger than 2.

\item pos(M) :- movieInfo(M, \_, 'comedy', 'R', Y, \_, \_), movieActor(M, 'Ryan Reynolds'), Y $>$ 2012

In natural language: A movie is profitable if it is R-rated and has 'Ryan Reynolds' as an actor and the movie came out later than 2012.

\item pos(M) :- director(M, 'Christopher Nolan', \_)

This example should be self-explanatory.
\end{itemize}

This approach is also useful for updating the database. If an actor wins an oscar only the entry in the actor table needs to be updated once. In a multi-instance representation all entries of the actor need to be changed. Furthermore, since the cast of a movie can be mapped from one movie to multiple actors a multi-instance approach would result in multiple table entries for one movie where only the actor name changes. This would be redundant and can be avoided using a relational representation.


\section{Representation for example 4 (Spam or ham): attribute valued representation}
For this example we use an attribute valued representation. For this we can use a table for multiple examples:

\begin{tabular}{|c|c|c||c|}
	\hline
	subject & knownSender & domain & class \\
	\hline
	supplements & no & other & spam \\
	money & yes & bank & ham \\
	other & yes & gmail & ham \\
	money & no & other & spam \\
	course & no & university & ham \\
	supplements & no & university & ham \\
	arabian prince & no & hotmail & ham \\
	\hline
\end{tabular}

\medskip
Possible hypthesis:
\begin{itemize}
\item ham :- spamorham(\_, \_, university) 

Meaning: We regard the email as non-spam when the sender domain is a university. (This actually seemed to be the case at my old university. We tried to get sent to the spam folder for a while. Nothing worked.)
\item ham :- spamorham(\_, yes, \_, \_)

Meaning: If we know the sender (e.g. from the contact book) we regard the mail as non-spam.
\item spam :- spamorham('arabian prince', no, \_)

Meaning: If the subject is about an Arabian prince and we do not know the sender, it is classified as spam.
\end{itemize}

We could also use an boolean representation, but then we would have to force all our attributes where we have more than two values into a boolean representation which could lead to a combinatorial explosion. If we reduce the granularity of the attributes to just boolean representations we could lose information.

I sadly do not know what a multi-instance representation would bring us here. But that could be due to the fact that I do not quite get what the major advantages over AV representation are. I also looked for explanations on the internet, but found nothing helpful (maybe I was too focused on this example).

\end{document}
